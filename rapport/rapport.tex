\documentclass[a4paper,11pt]{article}
\usepackage[T1]{fontenc}
\usepackage[utf8]{inputenc}
\usepackage{lmodern}
\usepackage{graphicx}
\usepackage{float}
\usepackage{enumitem}
\usepackage{pifont}
\usepackage[french]{babel}
\usepackage[a4paper, total={6in, 9in}]{geometry}
\usepackage{hyperref}
\setcounter{tocdepth}{4}
\setcounter{secnumdepth}{4}
\setlength{\parindent}{0in}
\renewcommand{\familydefault}{\sfdefault}

\hypersetup{
    colorlinks,
    citecolor=black,
    filecolor=black,
    linkcolor=black,
    urlcolor=black
}


\title{Compte rendu}
\author{
  Alexandre Legendre\\
  \and
  Eliott Sammier\\
  \and
  Vincenzo Carminati\\
  \and
  Elian Loraux\\
  \and
  Marion Chauvineau\\
  \and
  Lucas Lacouture\\
}

\begin{document}

\maketitle
\newpage
\tableofcontents
\newpage

\section{Cadrage du projet}
\subsection{Description du projet}
En cette période de crise sanitaire, et notamment durant le confinement, nous avons pu observer une
émergence de la solidarité entre voisins. Au travers d’un immeuble, d’une rue ou bien d’un quartier,
certains voisins ont œuvré pour une ambiance conviviale, voir familiale, afin d’apporter un soutien en
cette période. Au travers de concerts improvisés, de cours de sport ou même de variantes de jeux
télévisés “Question pour un champion” transformé en “Question pour un balcon”, un certain
communautarisme entre voisins a vu le jour. Récemment, les grandes crues dévastatrices dans le Sud-
Est ont soulevé un élan de solidarité. La finalité de notre projet serait de donner un cadre à cette
entraide pour pouvoir la rendre plus régulière et plus simple à mettre en place.\\

Notre projet est un service de mise en relation et d’échange de services entre particuliers à proximité.\\

Il prendra la forme d’un site Web. L’utilisateur pourra demander, consulter et répondre à un service.
Les services iront de la garderie aux petits travaux en passant par le dépannage informatique, ou
même des courses régulières. L’idée est de faire une plateforme participative et diversifiée grâce aux
compétences individuelles de chaque utilisateur.\\

Durant cette période de crise sanitaire, cela prend du sens notamment pour la mutualisation des
courses. Prenons l’exemple d’une personne fragile, à la place d’aller faire ses courses elle-même et
d’être par conséquent en contact avec tous les clients du magasin, elle n’est qu’en contact avec un
voisin. Inversement pour une personne atteinte ou avec des suspicions de Covid-19, cela limite les
contacts et par conséquent les risques de transmission.\\

\subsection{Objectifs}

Notre projet a pour but de simplifier l’entraide entre voisins. Cette simplification entraînera une
régularisation, une spontanéité et une diversité des services.\\

Cependant, cet objectif peut être diviser en plusieurs sous-objectifs.\\

%---------------==================== NOUVEAUX ====================-----------------

%----- ça c'est mes notes -----
%objectif générale du Projet = petit paragraphe

%faire une platforme d'echange entre particuler par Example
%qqoqcp + smart (post it?)
%spécifique -> particuler -- les vieux
%mesurable -> nombre de personnes
%temportel -> fin de projet ou contrainte en temps de reponse

%=>
%Facile d'utilisation (accecibilité W3C, FAQ...)
%Avoir des utilisateur de confiance (via certification et reputation)

%----- fin des notes -----
\underline{Facilité d'utilisation}\\

Un voisinage étant un des meilleurs exemples de mixité social.
On y trouve des jeunes, des personnes agées, des ouvrier, des ingénieurs, des familles, autant de personnes qui sont plus ou moins à l'aise en informatique.
C'est à nous de nous adapter à tous ses différents niveaux. Il faut donc que le site soit simple et facile d'utilisation.
Pour atteindre cet objectif, nous pouvons faire un guidage sur le site en suivant les règles d'accessibilité du W3C et faire une foire aux questions par exemple.\\

\underline{Des utilisateurs de confiances}\\

Selon les services on peut avoir besoin d'utilisateur de confiance.
Par exemple pour des travaux de plomberie, il faut quelqu'un capable de réparer et surtout de ne pas empirer les dégâts.
Ou encore, pour les personnes agées, il faut que les personnes qui interviennent au domicile ne soit pas mal intentionnée.
Il faut donc un moyen pour vérifier la confiance envers les utilisateurs.
% Je corrige les fautes après manger
Pour cela on a pensé à un système de réputation qui est un premier critère de confiance.
Autre possibilité qui peut venir en compléments est deux système de certification :\\
  - Un pour l'identité des utilisateur via un tier comme une mairie ou une association\\
  - Un pour les compétences des utilisateurs\\
le moyen n'est pas encore définie avec exactitude mais comme solutions envisagés il y a la certification par diplômes et/ou la certification de compétence par les usagers.
Si X personnes assure que la compétence utilisé lors du service rendu est aquise, la certification peut être donnée.
Il faut néanmois se protéger de ne pas certifier quelqu'un qui na pas les compétences.\\

%la suite au prochain push


%---------------==================== FIN NOUVEAUX ====================-----------------

\subsection{Description du contexte}

Il existe plusieurs sites fournissant des services similaires à ceux de notre projet, les principaux étant :\\

\begin{itemize}
  \item AlloVoisins qui propose un service de mise en relation entre particuliers avec 3,5 millions
  d'utilisateurs. La grande majorité des annonces y sont rémunérées, l’option non rémunérée
  n’étant que très peu utilisée. Certaines entreprises sont aussi présentes sur le site.

  \item Pwiic, c’est un site qui permet d’échanger des services ou objets auprès de particuliers. Il suffit
  d’envoyer un “Pwiic” pour que la demande s’affiche sur le site. Un algorithme se charge de
  l’envoyer aux bonnes personnes.

  \item YakaSaider, c’est un site communautaire qui met en relation des particuliers ayant besoin ou
  rendant un service. Il se base sur un nombre d’heure, pour chaque heure passé à rendre des
  services, l’utilisateur a le droit de demander une heure de service.
\end{itemize}

Nous avons comparé ces sites sur la base de critères simple (figure 1).

\begin{figure}[H]
  \includegraphics[width=\linewidth]{images/tableau-ergo.png}
  \caption{Tableau de comparaison des critères.}
  \label{fig:table1}
\end{figure}

La justification de ces choix se trouve en annexe.\\

\subsection{Acteurs du projet}
\subsubsection{Maîtrise d’ouvrage}

Le domaine d’activité visé est la mise en relation et l’échange des services pour particuliers.\\

Les enseignants sont ceux qui ont commandé le projet, par l’intermédiaire du sujet.\\

Les utilisateurs seront des personnes à risque, des personnes âgées ou simplement des particuliers
voulant échanger des services. Ils s’attendront à pouvoir demander et proposer des services
simplement à des personnes à proximité.\\

\subsubsection{Maîtrise d'œuvre}

Elle est composée 6 étudiants en 2 e année de DUT Informatique, qui s’occuperont de définir les
besoins, et de les réaliser.\\

Compétences attendues : être capable de réaliser une page web, la programmation côté serveur et
être capable de lier un programme à une base de données.\\

\subsubsection{Intervention de sous-traitants}

Les sous-traitants possibles sont des associations ou des mairies pour vérifier l’identité des utilisateurs
bénévoles.\\

\subsection{Contraintes}

Notre projet est affecté par l’environnement actuel, les conditions sanitaires, le public cible, etc...
Toutes ces inconnues définissent un certain nombre de contraintes.\\

\subsubsection{Contrainte temporelle}

Pour ce projet la contrainte temporelle est assez forte. En effet, le déroulé se présente sous forme de
3 itérations successives (le 13 Octobre 2020, le 24 Novembre 2020 et le 15 Janvier 2021) avec pour
chacune une date de rendu de dossier et d’évaluation.\\

La charge totale attendue est d’environ 950 heures et nous disposons de 17 semaines (vacances
comprises) pour réaliser ce projet.\\

\subsubsection{Contrainte sanitaire}

À la suite de la crise sanitaire du COVID-19, la moitié des séances de projet que nous avons sont en
distanciel et de manière générale il est recommandé de ne pas trop se regrouper.\\

Cela pose donc la question du travail à distance. En effet, une bonne partie du travail à fournir devra
être effectué à distance. Nous devrons utiliser des outils informatiques pour continuer à échanger des
informations, discuter de l’avancée du projet, répartir les tâches etc...\\

\subsubsection{Contrainte matérielle}

Le travail à distance implique aussi une contrainte matérielle. En effet, les membres du groupe doivent
avoir un minimum de matériel informatique pour pouvoir travailler depuis chez eux.\\

Les membres du groupe sont-ils tous dans de bonnes conditions pour travailler correctement à
distance ? Au minimum, il faudra un ordinateur pour développer le site, ainsi qu’une connexion
Internet.\\

\subsubsection{Contrainte juridique}

La création de compte sur le site implique le stockage des données personnelles de l’utilisateur. Ces
données pouvant être sensibles, telles que des adresses ou des mots de passe, elles sont soumises à
certaines lois. Nous devrons donc nous conformer à ces dernières, ce qui entraîne une contrainte
juridique.\\

\subsection{Risques}

Durant la réalisation ce de projet, il est possible que ce dernier ne s’exécute pas comme prévu. Il nous
faut donc tenter d’identifier ces risques et mettre en place un plan de mitigation.\\

\subsubsection{Risque temporel}

Une mauvaise organisation des membres du groupe peut entraîner un retard. Cette éventualité est
probable et son impact serait critique. En effet, nous pourrions accumuler ce retard, or les échéances
successives nous en empêchent. La criticité de ce risque est donc substantielle.\\

Une mauvaise compréhension des besoins est un évènement probable. Pour les mêmes raisons
qu'une mauvaise organisation, ce risque a un impact critique. Sa criticité est donc substantielle.\\

\subsubsection{Risque humain}

La fermeture des universités est un scénario probable pour les semaines à venir. Son impact serait
modéré car nous devrions majoritairement interagir en distanciel. Durant le dernier semestre, nous
avons appris à travailler à distance et nous avons les outils pour communiquer en distanciel. La criticité
de ce risque est donc modérée.\\

Si un ou plusieurs membres du groupe doivent se confiner. Ce scénario est probable. Mais son impact
serait majeur si la ou les personnes sont dans l’incapacité de travailler, ou mineur s’ils n’ont pas de
symptômes.\\

\subsubsection{Risque matériel}

La perte de documents est probable, une feuille égarée ou un disque dur défaillant sont des choses
qui peuvent arriver. Il est probable que ce risque survienne et son impact serait critique car le travail
devra alors être réalisé de nouveau, ce qui serait une perte de temps, une ressource précieuse dans
ce projet. Sa criticité est alors substantielle.\\

\subsubsection{Risque juridique}

L’utilisation du site pour des actes malveillants est une éventualité probable et son impact serait
majeur car cela pourrait mettre en danger les utilisateurs. Sa criticité est donc substantielle.\\

\begin{figure}[H]
  \includegraphics[width=\linewidth]{images/matrice-criticite.png}
  \caption{Matrice des risques.}
  \label{fig:matrice-risques}
\end{figure}

\subsubsection{Plan de mitigation}

Concernant la fermeture des universités ainsi que le confinement d’un ou plusieurs membres du
groupe, nous optons pour les réserves : nous allons mettre en place des outils supplémentaires pour
être capables de travailler correctement à distance.\\

Pour la perte de documents, nous choisissons la protection : nous mettrons en place une redondance
des documents en numérisant tous les documents physiques et en les stockant dans le cloud, ainsi
que sur le disque dur d’au moins deux de nos machines.\\

Enfin, par rapport à l’utilisation du site pour des actes malveillants, nous transférerons le risque à des
associations ou des mairies avec qui nous traiterons pour certifier les utilisateurs du site.\\

Concernant la mauvaise organisation du groupe, nous prenons la réduction. Nous nous organiserons
à l'avance et en se concertant tous ensemble. Si malgré cela l'organisation est mauvaise, nous
assumerons les pertes de temps. Quant à la mauvaise compréhension des besoins, nous choisissons
aussi la réduction. Nous poserons des questions à la maîtrise d'ouvrage et nous pourrions aussi sonder
les clients.\\

\newpage
\section{Expression du besoin}

L’utilisateur a soif d’une réponse à ces exigences.

\subsection{Besoins fonctionnels}

Le site devra permettre une mise en relation entre les particuliers (c’est sa fonction principale). Il
permettra ainsi à des individus proches géographiquement de se rencontrer pour pouvoir s’entraider
sur différents types de tâches.\\

\subsection{Base de données}

Les différentes informations que l'on connait sur chaque utilisateur (Pseudo, mot de passe,
adresse/localisation, qualifications, etc...) ainsi que ce qu'il a fait sur le site (Annonces postées,
annonces auquelles il a répondu, historique des messages, etc...) doivent être enregistrés même lorsque
l'utilisateur se déconnecte pour que le site fonctionne.\\

Il sera donc nécessaire de stocker toutes ces informations dans une base de donnée.\\

\subsection{Différents comptes}

Le site doit permettre la création de comptes pour les utilisateurs, mais également de comptes
pour les certificateurs, c'est à dire les organisations chargées de vérifier que les utilisateurs
qui proposent des services possèdent bel et bien les compétences requises pour proposer ces services.\\

Les comptes pour certificateurs permettant de certifier les autres comptes, une authentification
plus poussée devra être mise en place pour se connecter aux comptes certificateurs.\\

\subsection{Localisation}

Afin de pouvoir proposer à l'utilisateur uniquement les offres proches de chez lui, il faut que l'ont
puisse trier les annonces par leur localisation.\\

Il sera donc nécessaire, au moment de l'inscription d'un utilisateur, d'enregistrer sa localisation
géographique, afin de ne lui proposer que des annonces proches et de ne proposer ces annonces qu'à des
gens proches de lui.\\

On pourra récupérer la localisation de l'utilisateur par plusieurs moyens:\\
\begin{itemize}
  \item Simplement demander à l'utilisateur d'entrer son adresse.
  \item Récupéter la localisation de l'appareil via le navigateur.
\end{itemize}


\subsection{Besoins non fonctionnels}

Étant donné que le public visé par ce projet est très large, le site doit être accessible et utilisable par
tout type de personnes. Cela inclut donc les personnes qui ne sont pas familières avec Internet, ou
avec les ordinateurs en général. Il est donc nécessaire de rendre le site le plus accessible possible.\\

\subsubsection{Visuels adaptés}

Les textes devront être adaptés à l’utilisateur, par exemple si ce dernier a des problèmes de vue.\\

Il sera donc nécessaire de choisir des polices d’écritures simples à lire, de s’assurer que sa taille soit
réglable via le navigateur, d’utiliser un contraste et des couleurs adaptés.\\

Il faut également prendre en compte d’éventuels daltoniens qui pourraient utiliser le site, en
choisissant des couleurs assez différentes.\\

Pour cela nous nous vérifierons que les critères SQUARE de lisibilité et d’adaptabilité soient respectés.\\

\subsubsection{Avoir tout sous les yeux}

Beaucoup d’utilisateurs n’essaient pas de chercher plus en détails sur un site s’il ne tombe pas
directement sur ce qu’ils cherchent. Pour avoir une bonne qualité au sens SQUARE, le site doit avoir
une bonne capacité fonctionnelle, c'est-à-dire présenter les attributs et fonctions qui satisferont
directement le besoin utilisateur. Il faut donc réunir toutes les informations nécessaires à l’utilisateur
en haut de la page, de façon à ce qu’il ait accès à tout ce dont il a besoin sans avoir à défiler vers le
bas de la page. Il faut donc réussir à réduire la densité informationnelle, il faudra pour cela bien choisir
de quelles informations l’utilisateur a besoin sur chacune des pages, et les présenter de la façon la
plus brève et explicite possible. Le site pouvant être utilisé sur des appareils variés (ordinateurs,
téléphones, etc...), il faudra soigner l’adaptabilité et donc en particulier au niveau qualité SQUARE
garantir le critère de portabilité.\\

\subsubsection{Toujours savoir où aller}

Il est important que ce qu’on attend de l’utilisateur soit toujours évident, pour cela il nous faut un
guidage efficace. En effet, afin d’avoir un bon niveau de qualité au sens SQUARE, il est important de
garantir que la capacité fonctionnelle soit associée à une bonne facilité d’usage. Cela passe par la
facilité de compréhension de l’utilisateur. Il ne doit à aucun moment avoir à réfléchir à ce qu’il est
censé faire. Des textes et des boutons mis en avant sont nécessaires. Pour simplifier le guidage de
l’utilisateur, une gestion des erreurs et un retour sur le résultat de ses actions sont importants. On
devra également rester cohérent dans l’interface.\\

\subsubsection{Sécurité}

Les données que les utilisateurs nous confient, tels que leur adresse, numéro de téléphone, etc... sont
sensibles : une certaine sécurité doit donc être assurée.\\

\subsection{Fonctions attendues}

\subsubsection{Création d’annonces}

Notre site doit permettre aux utilisateurs de mettre en ligne des annonces créées par leurs soins. Cela
peut-être une annonce de recherche d’aide, ou une annonce pour au contraire proposer de l’aide. Le
site doit également permettre aux utilisateurs de personnaliser leurs annonces, en choisissant le type
de travail fourni/demandé, en ajoutant une description, et en précisant d’autres détails, tels que le
niveau d’expertise demandé ou laps de temps dans lequel le travail doit avoir été effectué.\\

\subsubsection{Recherche d’annonces}

Une page de recherche d’annonces sera disponible sur le site, et permettra aux utilisateurs de
rechercher des annonces correspondantes à leurs attentes. Ils pourront trier les annonces par
localisation, par type de service demandé, et par service proposé/demandé. Une fois que l’utilisateur
a trouvé une annonce qui lui convient, il pourra ensuite entrer en communication avec celui qui a
publié l’annonce via un chat textuel.\\

\subsubsection{Gestion de comptes}

Le site demandera aux utilisateurs de créer un compte avant de pouvoir créer ou répondre à des
annonces. Une page permettra de se connecter à son compte et une autre de gérer ses informations
personnelles.\\

\newpage
\section{Solutions}

\subsection{Spécifications détaillées}

Compte tenu des besoins détaillés précédemment, le service que nous souhaitons mettre en œuvre
sera réalisé par un service Web, et non pas une application logicielle à installer qui serait moins
adaptée au public cible.\\

\subsubsection{Maquette}

A titre d’exemple, voici deux pages qui préfigurent l’ergonomie et la charte graphique qui s’appliquera
à tout le site dont le reste des pages se trouvent en annexe. Les différentes pages doivent rester
simples en termes de composants, avoir des couleurs sobres et claires afin d’être compréhensibles
par le public cible, notamment des personnes âgées.\\

\begin{figure}[H]
  \includegraphics[width=\linewidth]{images/maquette-accueil.png}
  \caption{Page d'accueil.}
  \label{fig:maquette-accueil}
\end{figure}

\begin{figure}[H]
  \includegraphics[width=\linewidth]{images/maquette-recherche.png}
  \caption{Recherche dans une catégorie.}
  \label{fig:maquette-recherche}
\end{figure}

\subsubsection{Évaluation ergonomique}

Les principaux critères ergonomiques auxquels notre IHM va se conformer sont :\\

\begin{itemize}
  \item Le guidage : un fil d’Ariane toujours visible sur le haut de l’IHM permettra aux utilisateurs de
  se repérer facilement dans le site et sur les actions en cours. Une illustration des actions
  possibles et des formulaires sur le panneau central du site facilitera l’accès aux différents
  services.
  \item Le contrôle explicite : la suppression des offres de service sera protégée par une double
  demande et la mise en relation sera explicite afin de protéger les utilisateurs.
  \item La gestion des erreurs : un pop-up va être mis en place pour permettre un guidage des
  utilisateurs en cas d’erreur de saisie et en particulier pour aider les personnes âgées.
  \item La charge de travail : la création des offres se fera avec un formulaire afin de limiter les étapes
  nécessaires à l’enregistrement avec un pré-remplissage des champs lorsque cela est possible
  avec les informations de compte de l’utilisateur lorsqu’il est authentifié. La recherche des
  offres sera également simplifiée, avec l’utilisation des préférences de l’utilisateur pour
  préremplir les champs de recherche.
  \item L’adaptabilité : compte tenu du public visé, nous allons mettre en place un mode spécifique
  pour les personnes malvoyantes, activable avec une adaptation de la taille de la police pour
  une meilleur accessibilité.
\end{itemize}

\subsection{Solutions techniques envisageables}

La conception d’un site web dynamique se décompose en deux parties : d’une part l’IHM qui
s’exécutera sur le navigateur, et d’autre part la logique côté serveur.\\

Pour suivre un modèle MVC, l’application sera donc découpée en plusieurs parties, chacune ayant
besoin de solutions techniques différentes :\\

\begin{itemize}
  \item La logique applicative doit s’exécuter sur un serveur distant protégé qui garantira la protection
  des données des utilisateurs et gérera l’accès et le stockage persistant des données.
  \item La communication entre le serveur et le client se fera avec des protocoles standards pour la
  navigation web HTTP/HTTPS.
\end{itemize}

Ce modèle a pour avantage de permettre une maintenabilité accrue, par exemple si nous décidons de
changer l’affichage (la partie vue), nous n’avons pas à toucher à la logique (la partie modèle).\\

Pour implémenter la partie IHM, les solutions envisageables sont basées sur les différentes
technologies actuelles.
L’ossature des pages sera en HTML et du CSS pour gérer la charte graphique et la structure des
bandeaux et centres de page. On utilisera plus particulière la dernière version (HTML 5) qui permet
d’ajouter plus facilement des composants multimédias. Afin d’ajouter une dynamique moderne aux
pages (menu illustré par exemple) les sites web actuels utilisent également un framework Javascript
comme jQuery, AngularJS, ou React. Aux vues des délais et de la complexité des différent frameworks,
il est envisagé de n’utiliser que du HTML 5.\\

Pour implémenter la partie logique applicative et connexion aux stockages de données, qui doit être
sur un serveur distant sécurisé, il existe plusieurs technologies disponibles. On peut utiliser du PHP sur
serveur Apache pour s’interfacer avec des fichiers ou des bases de données. Il est aussi possible de
recourir à l’usage de conteneurs JEE comme Tomcat, ou des serveurs d’application comme Weblogic
ou JBOSS. L’usage de Java JEE offre plus de possibilités, mais du fait de l’intensité du travail demandé
pour le développement et la complexité associé à JEE, il est préférable de baser notre service sur PHP.
Pour implémenter le stockage de l’information, il existe plusieurs possibilité : le stockage dans des
fichiers ; ou bien l’utilisation de base de données, qu’elle soit non relationnelle comme MongoDB,
Cassandra, ou relationnelle comme PostgreSQL, Oracle DB, MySQL ou bien SQLite. Compte tenu des
informations à stocker et des possibilités de recherche offertes sur l’IHM, une base de données et un
modèle relationnel sont plus indiqués. Compte tenu du temps imparti et de la complexité de mise en
œuvre, il est envisagé d’utiliser une base relationnelle simple et facilement intégrable avec PHP et
dans un serveur Apache, comme SQLite ou MySQL.\\

Pour sécuriser l’accès aux services du site, il est nécessaire de mettre un place une authentification
avec un accès en mode HTTPS pour protéger les données qui circuleront sur le réseau. Il existe
plusieurs protocoles possibles pour l’authentification, notamment pour le SSO comme SAML ou
OAUTH2 avec certificats ou identifiant et mot de passe. En raison de l’échéance, il sera impossible de
mettre en place ces technologies. L’authentification se fera donc par un simple identifiant et mot de
passe sur HTTPS. Les informations d’identification seront stockées dans une partie spécifique de la
base de données. Cette dernière stockera aussi les préférences des utilisateurs.\\

\subsection{Réalisations techniques}

La conséquence des choix techniques envisagés pour la réalisation : HTML/CSS, Apache, PHP, MySQL.
Il va être nécessaire de mettre en place un serveur Apache avec son module PHP et le driver MySQL
associé sur un serveur Linux.\\

Pour le développement nous utiliserons des éditeurs de code standard, et GIMP pour le design des
images et composants graphiques.\\

Le partage du code, le stockage et la gestion des versions seront gérés par un dépôt Git hébergé sur
Gricad-GitLab.\\

\subsection{Organisation du travail}

\subsubsection{Choix du modèle}

Pour ce projet, nous utiliserons une méthode itérative. Nous effectuerons des itérations toutes les 4
à 5 semaines de travail.\\

\subsubsection{Découpage du projet}

Pour la deuxième itération, le projet a été découpé en 3 parties. L’équipe sera divisée en 3 binômes.\\

Ce choix de binôme permet une plus grande flexibilité et une parallélisation des tâches, tout en étant
moins stressant que du travail seul.\\

Sur ce diagramme, Alexandre et Lucas seront les verts,
Elian et Marion seront les bleu clair et
Eliott et Vincenzo seront les bleu foncé.\\

\begin{figure}[H]
  \includegraphics[width=\linewidth]{images/gantt-iteration2.png}
  \caption{Diagramme de GANTT pour l'Itération 2}
  \label{fig:gantt-iteration2}
\end{figure}

\begin{figure}[H]
  \includegraphics[width=\linewidth]{images/gantt-iteration3.png}
  \caption{Diagramme de GANTT pour l'Itération 3}
  \label{fig:gantt-iteration3}
\end{figure}

%---------------==================== NOUVEAUX ====================-----------------
\newpage
\section{Conception de la solution}

\begin{figure}[H]
  \includegraphics[width=\linewidth]{../Conception/DCU.png}
  \caption{Diagramme des cas d'utilisation}
  \label{fig:<un-label-court>}
\end{figure}

\subsection{MVC}
\subsubsection{Modèle}
Le modèle est composée de plusieurs classes PHP comme définie par le diagramme de classe ci-dessous
plus la classe DAO qui permet l'acces a la base de données qui n'apparait pas pour des question de visibilité :\\

\begin{figure}[H]
  \includegraphics[width=\linewidth]{../Conception/PHP/DC.png}
  \caption{Diagramme de classe php}
  \label{fig:<un-label-court>}
\end{figure}

De même que pour une question de lisibilité les getteur n'apparaise mais il y a un getteur pour chaque attribut.

%---------------==================== FIN NOUVEAUX ====================-----------------

\newpage
\section{Annexe}

\subsection{Analyse de l'existant}

\subsubsection{Services}

Notre projet consistant à fournir un service, il nous fallait regarder quels sont nos concurrents sur le marché.
Visent-ils une rentabilité ou juste à rendre service ? Pour cela, nous avons vérifier si, pour chaque site,
il était possible de rendre un service gratuitement, contre un autre service ou en étant payé.\\

AlloVoisins possède des services gratuits (dits “non-rémunérés”) mais ils sont très peu utilisés,
dû au fait qu’il est impossible d’échanger ce service contre un autre.\\

\begin{figure}[H]
  \includegraphics[width=200px]{images/services-allovoisins.png}
  \label{fig:services-allovoisins}
\end{figure}

Pwiic propose un système d’échange de pièces contre des services. Ces pièces peuvent être soit achetées,
soit gagnées en échange des services. Le site se base sur une proposition de prix, plutôt qu’un prix fixé au départ.\\

\begin{figure}[H]
  \includegraphics[width=250px]{images/pieces-pwiic.png}
  \label{fig:pieces-pwiic}
\end{figure}

YakaSaider met en place des profils : par exemple, ici Aurélie est coiffeuse. Il n’y a aucun prix affiché,
aucune durée, juste un bouton Contacter. Sur ce site, il faut payer en heures. Lorsque l’on crée un compte,
on dispose d’une heure “gratuite” après quoi il faut rendre des services pour en gagner.
On les consomme lorsque l’on demande un service.\\

\begin{figure}[H]
  \includegraphics[width=500px]{images/aurelie-yakasaider.png}
  \label{fig:aurelie-yakasaider}
\end{figure}

\subsubsection{Critères d'ergonomie}

Étant donné le public cible, il est important de respecter certains critères ergonomiques.\\

\paragraph{Guidage}

Nous nous sommes laissés guider sur les sites, pour voir comment ces sites s’y prenaient pour attirer
l’attention là où se trouve l’information.\\

AlloVoisins a un guidage simple. Les boutons importants sont en couleur.\\

\begin{figure}[H]
  \includegraphics[width=350px]{images/Guidage-allovoisins.png}
  \label{fig:Guidage-allovoisins}
\end{figure}

Mais pour les demandes, le titre de la demande est l’endroit où il faut cliquer pour accéder aux informations,
or il n’est pas assez mis en valeur.\\

\begin{figure}[H]
  \includegraphics[width=350px]{images/demande-allovoisins.png}
  \label{fig:demande-allovoisins}
\end{figure}

Pwiic a un guidage peu intuitif. Par exemple, sur les cartes servant à donner les caractéristiques d’une personne
lorsque l’on cherche un service, le bouton “Voir ce Pwiic” n’est pas en couleur. De plus, le pouce en l’air donne
l’impression que le bouton est là pour donner un avis et non pour voir le profil.\\

\begin{figure}[H]
  \includegraphics[width=350px]{images/guidage-pwiic.png}
  \label{fig:guidage-pwiic}
\end{figure}

Et lorsque l’on clique sur Voir ce Pwiic, le bouton pour contacter la personne n’est pas trouvable du premier coup d’œil.\\

\begin{figure}[H]
  \includegraphics[width=350px]{images/voir-ce-pwiic.png}
  \label{fig:voir-ce-pwiic}
\end{figure}

De plus, le mot “Pwiic” est un terme abstrait créé à partir du nom de marque du site, moins clair que “annonce”
par exemple, ce qui ne facilite pas forcément la navigation pour des personnes peu à l’aise.\\

YakaSaider possède un guidage peu efficace. Lorsque l’on recherche pour “Tous les services”, on ne voit aucune
compétence ni même description, seulement le visage de la personne, son nom et sa ville.\\

\begin{figure}[H]
  \includegraphics[width=400px]{images/guidage-yakasaider.png}
  \label{fig:guidage-yakasaider}
\end{figure}

Pour régler le problème, il faut faire une recherche plus affinée.\\

\paragraph{Gestion des erreurs}

Ensuite, nous avons essayé de faire des actions qui n’étaient pas celles demandées, une erreur peut très vite perdre un utilisateur.\\

AlloVoisins et Pwiic semblent bien gérer les erreurs. Par exemple, si l’on saisit dans la barre de recherche quelque chose
qui n’est pas valable, on retourne sur la page de résultats par défaut, c’est à dire tous les résultats près de
chez nous ou de l’endroit où notre demande se trouve.\\

\begin{figure}[H]
  \includegraphics[width=\linewidth]{images/gestion-erreur-allovoisins.png}
  \label{fig:gestion-erreur-allovoisins}
\end{figure}

\begin{figure}[H]
  \includegraphics[width=400px]{images/gestion-erreur-yakasaider.png}
  \label{fig:gestion-erreur-yakasaider}
\end{figure}

YakaSaider gère le résultat de la recherche différemment. Lorsqu’aucun résultat n’est disponible,
il affiche simplement un message disant qu’aucun résultat n’a été trouvé.\\

\begin{figure}[H]
  \includegraphics[width=\linewidth]{images/pas-trouve-yakasaider.png}
  \label{fig:pas-trouve-yakasaider}
\end{figure}

Mais la liste des services possède une erreur, on peut sélectionner les tirets qui servent à délimiter les catégories.\\

\begin{figure}[H]
  \includegraphics[width=\linewidth]{images/erreur-yakasaider.png}
  \label{fig:erreur-yakasaider}
\end{figure}

\paragraph{Certification}

Enfin, la certification nous semblait importante, puisqu’elle permet de limiter certains risques, tels que les arnaques.\\

YakaSaider et Pwiic ne possèdent aucun système de certification.\\

Pwiic ne demande même pas de nom, l’utilisateur s’identifie par un pseudo.\\

AlloVoisins possèdent un système où un utilisateur est certifié par son numéro de téléphone,
cela lui permet notamment d’accéder à la messagerie du site. (\href{https://support.allovoisins.com/hc/fr/articles/360000816614-Pourquoi-dois-je-certifier-mon-num%C3%A9ro-de-mobile-pour-pouvoir-acc%C3%A9der-%C3%A0-la-messagerie-}{source})

\subsection{Maquettes}

\end{document}
